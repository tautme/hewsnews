\documentclass{article}

\usepackage[utf8]{inputenc}
\usepackage[T1]{fontenc}
\usepackage{microtype}

\usepackage{newspaper}

\date{\today}
\currentvolume{1}
\currentissue{1}

%% [LianTze] The newspaper package also provides 
%% these commands to set various metadata:

%% The banner headline on the first page
%%   (The colon after s: is to get a more
%%   modern majuscule s in this font instead of 
%%   the medieval tall s. For anyone interested 
%%   in the history: 
%%  http://medievalwriting.50megs.com/scripts/letters/historys.htm)
\SetPaperName{Covered Way}

%% The name used in the running header after
%% the first page
\SetHeaderName{The Covered Way}

%% and also...
\SetPaperLocation{Smackover Arkansas}
\SetPaperSlogan{``We write what matters.''}
\SetPaperPrice{Zero Dollars}


% [LianTze] times (the package not the font) is rather outdated now; use newtx (see later)
% \usepackage{times}
\usepackage{graphicx}
\usepackage{multicol}

\usepackage{picinpar}
%uasage of picinpar:
%\begin{window}[1,l,\includegraphics{},caption]xxxxx\end{window}


%% [LianTze] Contains some modifications
\usepackage{newspaper-mod}
%%... so now you can redefine the headline and byline style if you want to.
%% These can be issued just before any
%% byline or headline in the paper, to
%% individually style each article
%%
% \renewcommand{\headlinestyle}{\itshape\Large\lsstyle}
% \renewcommand{\bylinestyle}{\bfseries\Large\raggedright}


%%%%%%%%%  Front matter   %%%%%%%%%%

\usepackage{lipsum}

\begin{document}
\maketitle

\begin{multicols}{3}

\byline{
% \LaTeX{} 
One two three boom
% Let me think about it; give a minute
 }{Micelle Reposa}
 
Room to drag me under the bus so fast you. Shaving shaving beard.  The purpose is to love the poor love Birdy.  Is to love the poor it has nothing to do with whether or not you are poor it doesn't have to do with whether you're already loving the board I love the poor.  Show gods love because a lot of them have not seen God's love.  Whatever. Or a car crash.  
You just asked me six questions and then I had answers to the first five and then I was trying to remember them all but then the last question is I'm still working on the questions and so it's a little difficult.  Let me tell you the first question which was gonna roll out yeah you do sound better, is the earlier question about the poor and then you said what are you and Annie go home and do y'all sit there and talk about like what that crazy person was doing and I will look at that well let me regurgitate my wisdom because people say I'm always right and very wise is that I would say no I don't do that because that would be gossiping.  

%The package is basically a redefinition of the \verb+\maketitle+ command.  The model was the New York Times---hopefully I haven't violated any copyright laws.  I also had to redefine the plain pagestyle.  It kept me busy for a few nights after work.  The rest is packages other people have written.

The twenty eighth day of May in the year of twenty and twenty one was a near summer day with the coolness and sunshine that makes a day worth spending outside.  



%\begin{window}[2,r,\includegraphics[width=1.0in]{atom.jpg},\centerline{The Atom}] The \verb+multicol+ package allows using multiple columns without starting a new page.  Using floats is not possible in a columns environment, however with the \verb+picinpar+ package, I can set a picture inside a block of text---just like you one you see here.  Isn't \LaTeX{} cool?
%And now we're just filling more space, and yet more space.  
%\end{window}

\closearticle


\headline{Adam Hughes News}
%This is just an example to fill up some space, but as long as I have your attention, I'll give some newspaper advice.
Don't talk about my friend like that. This is the time we have and we are not guaranteed any other moment. Do we get agitated about. Yeah because that's the way of the world and it feels good in the moment but we don't really know the true pleasure of also being in connection with God but once we get a taste of that then we realize like wait a minute all that stuff that he was talking about don't do this don't do that 10 Commandments this content of my meds that even the stuff that doesn't make sense it ends up making sense because we start.  

We start to see the process where oh wow that I didn't do this in this in this this this this this and then after a long time practicing those things if you see the connection where it goes all the way around and it flips through all these different dimensions and it's like a mystery of how it even happened but then you come back to the beginning and you just said yourself well I wanna follow God's commands because even though I don't know how it happened I know what the outcome is and I know what it does feel like to be closer to God. Keep on practicing that and it is a practice just like you asked me do the thoughts come to you and then you have to review them it's not like oh it's a one and done it's like oh after I do it and I do it a practice and get better and better at it and it just keeps getting better and better in the mysterious ways that happens I don't know what's going on but I sure do know I like the end result of being in communion with God and and how that relates totally to the things we experience in this world all the heavenly stuff like that. .Being with people who are also in that zone and it's like once you're with them and you experience that there's nothing in this world you trade for gold silver Ferraris you sacrifice at all and then you stop caring about the material stuff you stop caring about the wealth in the money and the food and the pleasures and we realize the time we have with people is the most important thing and so we cultivate it and we keep doing it and it's almost like.  

I've experienced where I think to myself and this is relevant to right now me and you where I know what's healthy for me to eat and I know like if somebody cared about me well and they had to fix my food will they fix the food the certain way and they feed it to me and it's like the food I don't even like all the sudden I wanna eat it just because it's healthy and just because I know God wants me to be healthy and like the stuff I used to eat when I was in that mindset like now I go back and I'm like.  And get water even though I really want to Coke how did I go to the the the the the gravy covered chicken fried steak in choose oatmeal to eat and it was like the reason was going all the way back to the time we have with people it's like the food doesn't matter it's the time that I'm getting to spend eating the food with the person I'm with it weird.  

%I suppose we could also show how an equation is type set:

%\begin{displaymath}
%x=\frac{-b\pm\sqrt{b^2-4ac}}{2a}
%\end{displaymath}

%and there you have it.  

%\lipsum[1-4]

\closearticle


\headline{Wall Street with Wallace}
Explain what just happened no offense but your last expose what 
you went through it went through this popularity contest relinquished 
I also how do you say preface and practicing and practicing this 
with no take no offense.  Don't let the sun so if you too much but 
your last expose I found found offensive because you just went 
through this whole popularity contest thing when popularity 
doesn't mean anything and then you said I'm hanging out with 
people because I want to be around popular people and then 
you said you want to be popular and have all these times.  
Away from the world and the worlds ways of doing that weird 
stuff.  And that's exactly what I just said was a little 
offensive because you saying that I'm shallow enough to to to 
to not introduce somebody say something because they're not 
popular in and all that stuff oh Michelle.  

Oh and scholarly mustard someone with the connection.  Temple here's 
a very good example that will take your expose and expose that it 
into a pancake do you think Zack is popular in a fancy with all his 
stuff has he ever met my sister no I just told you.  Because I was 
correct then I would've already introduced him to my sister squish 
pancake flat.  Geniuses that I'm usually right and I'm very wise.  
Play it with a hint – dump truck load of sarcasm.  
\closearticle

%\headline{Memorial Day 2021}
%
%\begin{window}[2,r,\includegraphics[width=1.3in]{Raising_the_Flag_on_Iwo_Jima,_larger_-_edit1.jpg},\centerline{Raising the Flag}] On Friday morning, February 23, 1945, four days after the Marines landed at Iwo Jima, Joe Rosenthal was making his daily visit to the island on a Marine landing craft when he heard that an American flag was being raised atop Mount Suribachi, a volcano at the southern tip of the island. Three of the men in the photo where later killed in fighting on the island.  Three of the men in the photo where later killed in fighting on the island.  Three of the men in the photo where later killed in fighting on the island.
%\closearticle

%\headline{December 23, 1944}
\headline{Battle of the Bulge}
23rd of December 1944 mission lost 10 B-26's and was awarded the 
Presidential Unit Citation for the 397th Bomb Group. Target: Eller 
Railroad Bridge
%Nord de Guerre

\closearticle


\headline{Capt. Mont F. Stephensen}
%http://www.miaproject.net/mia-search-recoveries/hunconscious/

It was about 09:10 on this cold Saturday morning, December 23, 1944, 
when the last B-26 bomber left the base at Péronne France. The 4000hp 
of the two Pratt and Whitney R-2800 engines painfully lifted 
``Hunconscious”, the B-26G piloted by First Lieutenant William P. 
Cook of Alameda, California. Within minutes, she climbed to her 
cruising altitude and joined thirty-three other planes of the 
397th Bomb Group for the day’s mission, an attack against a railroad 
bridge spanning the Moselle River at Eller, Germany.

The first plane lost was ``Hunconscious”, Lieutenant Cook’s ship. 
Hit in the right engine, torque from the good motor pulled the 
machine into a snap roll. It then plummeted down toward earth. 
Cook and five other men perished.

Seconds after Flak hit Cook, a shell struck Lieutenant 
Charles W. Estes ship, ``Bank Nite Betty”. The deputy box 
leader and his crew described the incident during their 
post-mission debriefing. ``Three minutes after Malmedy, aircraft 
144 hit at about top turret and split in half. … No chutes seen.” 
All seven men aboard Estes’s aircraft died.

The group flew in a two-box formation and followed a pathfinder 
aircraft and three ``Window” aircraft. The latter carried 
no bombs and had the task of releasing strips of aluminum 
chaff to disrupt radar-controlled antiaircraft guns. The 
pathfinder carried four 500-pound bombs. The other planes 
each carried four 1,000-pound bombs. They flew for an hour 
over friendly territory before reaching the frontline city 
of Malmedy, Belgium. The airmen circled the city and took 
a course toward their objective. Enemy antiaircraft batteries 
immediately engaged the formation which was visible from the 
ground. At 10:20 AM, German gunners scored two hits. Sergeant 
McGinnis, a top-turret gunner, described the results. 
``Right off the bat, Flak got a ship in the right engine. 
She went over on her back. … Another blew in half.” 
Both victims came from Box II, the trailing box.

After losing Cook and Estes over Belgium, the 397th pressed ahead into German airspace and began the bomb run despite continued Flak. The aircraft jettisoned their explosives at 10:37 AM, bombing on the pathfinder’s mark. They hit the bridge. The formation made abroad left turn and headed toward home. Then disaster descended. Luftwaffe fighters pounced on the Americans. Aerial combat raged for ten minutes before P-47s intervened and broke up the attack. Flak and fighters claimed eight bombers over Germany. Only five planes returned to France without damage. Four of them were no longer airworthy and requiring months to repair. December 23, 1944, constituted the worst single-day losses for the 397th: ten aircraft downed, forty-five men dead, and twenty-one men prisoners of war. The group later received a Distinguished Unit Citation for destroying the Eller bridge.

After hostilities ended, the American Graves Registration Command accounted for thirty-nine of the dead. Lieutenant Cook and his crew were the only ones missing. The American Battle Monuments Commission later inscribed their names on the Wall of the Missing at the Luxembourg American Cemetery.

Decades later, during the 1990s, German aviation researchers began investigating the loss of Cook’s aircraft. The missing aircrew report contained little information regarding the circumstances of its disappearance. The report offered speculation that the aircraft fell somewhere in the geographic area where Luftwaffe fighters attacked the bombers. After years of work, they pinpointed nine 397th crash sites. Only one was in Belgium. This was the crash site of “Bank Nite Betty”, Estes’s aircraft. The plane had broken in two pieces while at altitude. The forward fuselage landed in a cow pasture southeast of Allmuthen, Belgium. The two German researchers thought the tail section had landed nearby, a belief that would eventually prove false.

During their research, they determined that antiaircraft fire had struck two 397th aircraft over Belgium, and both went down. The researchers knew that Estes had piloted one of them. Who piloted the other one? There was only one possibility—Cook. But where exactly did he crash?

In the autumn of 2006, the picture became clearer thanks to a German forestry worker affiliated with an organization called the Airwar History Working Group Rhine-Moselle. He had learned about the spot where the tail of Estes’s aircraft was supposed to have crashed. There was an impact crater. He searched in the vicinity and discovered over one hundred pounds of aircraft wreckage and shreds of U.S. Army Air Force clothing, including a leather fragment from the collar of a B-3 winter flying jacket. It bore a laundry mark, H-7489.

The Air War History Working Group identified the wreckage as being from a B-26. More importantly, they determined the laundry mark had no relation to anyone aboard Estes’s aircraft. The mark instead related to Cook’s bomb toggler, Sergeant Eric M. Honeyman (ASN 39037489) who also hailed from Alameda, California. This revelation led to an expanded search at the site in November. British-born aviation researcher Danny Keay joined the effort, and the group discovered more artifacts and two long-bone fragments.

Keay and the Germans contacted the U.S. Army Memorial Affairs Activity-Europe (USAMAA-E). In December 2006, two representatives from that organization arrived on site, conducted an investigation and took possession of the bone fragments and other artifacts. USAMAA-E deferred the case to the Joint POW-MIA Accounting Command (JPAC) for immediate action. Two JPAC historians visited the site in early 2007 and reported there were enough evidences “to warrant further investigation.”

Years rolled by. Despite the discovery of Honeyman’s flying jacket and meaningful bones, no recovery operation occurred. History Flight, a US non-profit organization, became involved in 2011 after developing a relationship with members of the 99th Division MIA Project. Bill Warnock, the project leader and his Belgian team mates Jean-Louis Seel and Jean-Philippe Speder had twenty years experience searching for missing soldiers of the 99th Infantry Division, having recovered twelve of them. The crash site lay on the periphery of the 99th sector and was well known to the Belgian-American group.

History Flight partnered with the group, and together they reconnoitered the crash site in June 2011 and found the area untouched since 2006. The task was going to be difficult. The site had been harvested years before and was now covered with a bush like vegetation. Moreover, the crater had been filled with branches and discarded logging cuttings. Seel and Speder extended the search on site in the Fall and unearthed many meaningful evidences, including equipment worn by crew members but no remains surfaced. This led to a larger operation in April 2012 which History Flight funded. Under the guidance of three American archaeologists, the consortium surveyed the site, pulled forest debris from the crater and drained its water for the first time in over sixty years. Buster, a cadaver dog attached to the Mammoth Lake Police Department in California, was also present to detect the potential presence of human decomposition chemicals. The team found more evidence identifying the specific B-26 model, a model consistent with Cook’s ship and not Estes’s. The team also recovered human remains and reported this to JPAC, USAMAA-E, and the Defense Prisoner of War/Missing Personnel Office (DPMO).

JPAC had now no other option than facing up to its responsibilities and dispatched forensic anthropologist Derek Benedix to the site. He examined the remains and recommended that JPAC conduct an immediate excavation of the crater and adjacent terrain. The first JPAC team arrived on site in May and began what proved to be a huge recovery operation. Beside the crater and its immediate surroundings, Honeyman dogtag editedthe debris field extended deep into a forest. A sizable quantity of human remains were eventually recovered as well as personal belongings, all consistent with the missing crewmen.The first element to surface was Sgt Honeyman’s dogtag. More were unearthed as the number of excavation Watch 01units augmented. Cook’s ID bracelet and dogtag, Swallwel and Lane’s dogtag and LeFavre’s bracelet and shoes were recovered. Unidentified but meaningful objects also came to light. Beside  a Sterling silver pilot wing badge in pristine condition, a torn officer’s cap badge,  English and French coins, a lighter, a fountain pen were among personal effects. Of particular significance an intact Elgin manufactured A-11 watch that had frozen the time of the crash. 10:25.

ong before thWard-C-Swalwel-Jre Belgian-American coalition started to work on this project, they had established contact with family members of Lieutenant Cook’s crew, including the sister of Ward Swalwell, a Chicago native and waist gunner aboard “Hunconscious”. When Jackie learned that her brother’s probable resting place had been located, she decided to visit the spot with her family. Escorted by 99th MIA Project members, they toured the site, met with JPAC team, and attended an improvised memorial service at the site. JPAC’s mission ended on September 17, 2013 after six recovery teams rotated on site. A total combined surface area of approximately 6400 square yards was excavated to an average depth of one foot.

In a memorandum dated July 22, 2014, JPAC confirmed the positive identification of the six crew members and notified the families.

Maurice J. Fevold was buried in Badger, Iowa, with full military honors on October 20, 2014. More than four miles of flags lined both sides of the street leading to Blossom Hill Cemetery for the soldier’s funeral procession.The Badger Fire Department also parked fire trucks along the route, and the Iowa Air National Guard provided a KC-135 plane that flew over the service.

William Parker Cook, the 27 year old aspiring surgeon and pilot of Hunconscious, had left his medical internship to enlist in the Army Air Forces. Just before going overseas, he had married his fiancee, Jean Swanson. He was buried in Mountain View Cemetery at Oakland, California, on October 26, 2014.

Frank G. Lane, Jr found his final resting place in Willoughby, Ohio, on May 2, 2015.

Eric M. Honeyman, although raised in California and enlisted in the Army Air Forces, was of Canadian origin. He was not granted to access US citizenship because he had not reached the legal age of 21. Eric’s parents, Bella and Eddy, had both donated their bodies to science, so there is no graves. The family decided to bring him home. Eric Honeyman was buried next to his grand parents in Trail, British Columbia, Canada, on June 22, 2015.

The last two crew members, co-pilot Arthur J. LeFavre and radio gunner Ward C. Swalwell, Jr were buried in Arlington National Cemetery, Virginia. LeFavre and a second casket containing unidentified osseous remains of the crew members were buried on August 18, 2015.  Ward Swalwell was buried next to them two days later, on August 20, 2015.

99th Division MIA Project members William C Warnock, Jean-Philippe Speder and Jean-Louis Seel,  History Flight, represented by Paul Schwimmer and his wife Audrey, joined the families and relatives of the six crew members. After a short funeral service held at the Old Post Chapel of Fort Myers, the flag draped casket with the unidentified remains of the six men was carried to the grave site by an artillery caisson drawn by six magnificent white horses. The horse escort and the caisson are reserved for officers and Medal of Play videoHonor recipients. LeFavre and Swalwell were placed in hearses and transferred to the grave site. After a brief eulogy, casket teams of the « Old Guard » folded the American flags in triangle with their unique, perfect and ceremonial drill. The chilling notes of « Taps » and a three rifle volley concluded the ceremony.

Abandoned in a Belgian forest for nearly seventy one years, Lt Cook and his crew found a final resting place befitting with the measure of their sacrifice.

Sources:
MACR 11985
Lt Cook, F/O LeFavre, S/Sgt Swalwel – IDPF’s
S/Sgt Lane, S/Sgt Fevold, Sgt Honeyman – IDPF’s
Preliminary report – Danny Keay



``A mormon from salt lake city, gosh... he was a great guy, 
he was killed later on, o gee... He was my buddy. You know, 
your lucky if you end up in life with one real buddy and I 
did that, and he was it." - 1st Lt. Jack L. Turner
Aircraft 4434

%https://b26.com/guestbook/2002.htm
%Date:
%5/5/2002
%Time:
%10:55:48 PM
% 
%Mont Fermore Stephensen BombGp: 397 Squadron: 599 Years: 1943-44 Location: Gardner Field, Calif. I am looking for information about my uncle Mont Stephensen. Lost in action December 23, 1944. The picture on Andy Andersons page is the same one that I have in my home. Thanks for your help. Mont

%Date:
%5/7/2002
%Time:
%8:17:10 AM
% 
%My brother, Capt. Mont Fermore Stephensen was a pilot, and squadron commander of a B26. We are trying to secure information to contribute to the Confederate Air Force Museum in Mesa, Arizona. Mont and his crew went down over Germany on Dec. 23, 1944. His plane was Draggin' Lady No. 3. 
% 
%Crew members were Lt. Elmer R. Borden, Lt. Laverne F. Grundman, St. Robert J. Kinney, St. John L. Grapes, T/Sgt. William E. Bower, S/Sgt. Lynn E. Rose, Jr., S/Sgt. Harry H. Watson and S/Sgt. James B. Galati.

\begin{window}[2,r,\includegraphics[width=1.3in]{43-34434-Page-4-copy.jpg},\centerline{Eller Railroad Bridge}] Mont Fermore Stephensen had enlisted in the United States Army Air Forces. Served during World War II. Stephensen had the rank of Captain. His military occupation or specialty was Pilot. Service number assignment was O-727795. Attached to 397th Bomb Group, 599th Bomb Squadron. \#43-34434 incident
%%https://www.honorstates.org/index.php?id=388508
%
%43-34430, 397BG, 599BS, "HUNCONSIDIOUS", Code 6B-B, Missions flown 4, MARC 11985
%1.Lt. William P.Cook; F/O Arthur J Lefavre; Sgt Eric M Honeyman; S/Sgts Ward C Swalwell, Jr; Frank G Lane, Jr; Maurice J Fevold (All KIA). 
%%https://b26.com/page/marauder_missions_december_23_1944.htm
\end{window}

\closearticle


\headline{1.Lt. Robert E. McCarthy}
Robert E McCarthy is buried or memorialized at Plot E Row 4 Grave 38 Luxembourg American Cemetery Luxembourg City, Luxembourg. This is an American Battle Monuments Commission location.
Aircraft 4139
%43-34139, 397BG, 599BS, ``TOOT SWEET'', Code 6B-Z, Missions flown 36, MARC 11986 
%15 Aug 44 to 23 Dec 44 shot down, circumstances unknown. 1.Lt. Robert E. McCarthy; 2.Lt's Charles F Abel; Harold T Ptaskiewicz; T//Sgt Clinton B Mathews; S//Sgts Ellis J Williamson; David E Hyre. (McCarthy, Ptaszkiewicz, Matthews, KIA; rest of crew, POWs) 
%Ellis J. ``Willie'' Williamson, aka ``Mr. Willie'' tells the story here on internet video: ``Gallery Furniture - Mattress Mack Sits Down With World War II Veteran Ellis J. Williamson'' 24 minute interview.
%https://www.youtube.com/watch?v=jYhApho19ro

%http://www.polkcountytoday.com/yauponcover121419.html
%https://www.youtube.com/watch?v=63AoIEijbhs
\closearticle

%\end{window}

%\closearticle

\end{multicols}

\end{document}
