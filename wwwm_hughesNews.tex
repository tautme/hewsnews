%\documentclass[legalpaper, 10pt]{article}
\documentclass{article}

\usepackage[utf8]{inputenc}
\usepackage[T1]{fontenc}
\usepackage{microtype}

\usepackage{newspaper}

\date{\today}
\currentvolume{1}
\currentissue{1}

%% [LianTze] The newspaper package also provides 
%% these commands to set various metadata:

%% The banner headline on the first page
%%   (The colon after s: is to get a more
%%   modern majuscule s in this font instead of 
%%   the medieval tall s. For anyone interested 
%%   in the history: 
%%  http://medievalwriting.50megs.com/scripts/letters/historys.htm)
\SetPaperName{Covered Way}

%% The name used in the running header after
%% the first page
\SetHeaderName{The Covered Way}

%% and also...
\SetPaperLocation{Smackover Arkansas}
\SetPaperSlogan{``We write what matters.''}
\SetPaperPrice{Zero Dollars}


% [LianTze] times (the package not the font) is rather outdated now; use newtx (see later)
% \usepackage{times}
\usepackage{graphicx}
\usepackage{multicol}

\usepackage{picinpar}
%uasage of picinpar:
%\begin{window}[1,l,\includegraphics{},caption]xxxxx\end{window}


%% [LianTze] Contains some modifications
\usepackage{newspaper-mod}
%%... so now you can redefine the headline and byline style if you want to.
%% These can be issued just before any
%% byline or headline in the paper, to
%% individually style each article
%%
% \renewcommand{\headlinestyle}{\itshape\Large\lsstyle}
% \renewcommand{\bylinestyle}{\bfseries\Large\raggedright}


%%%%%%%%%  Front matter   %%%%%%%%%%

\usepackage{lipsum}

\begin{document}
\maketitle

\begin{multicols}{3}

\byline{
% \LaTeX{} 
One two three boom
% Let me think about it; give a minute
 }{Micelle Reposa}
 
Room to drag me under the bus so fast you. Shaving shaving beard.  The purpose is to love the poor love Birdy.  Is to love the poor it has nothing to do with whether or not you are poor it doesn't have to do with whether you're already loving the board I love the poor.  Show gods love because a lot of them have not seen God's love.  Whatever. Or a car crash.  
You just asked me six questions and then I had answers to the first five and then I was trying to remember them all but then the last question is I'm still working on the questions and so it's a little difficult.  Let me tell you the first question which was gonna roll out yeah you do sound better, is the earlier question about the poor and then you said what are you and Annie go home and do y'all sit there and talk about like what that crazy person was doing and I will look at that well let me regurgitate my wisdom because people say I'm always right and very wise is that I would say no I don't do that because that would be gossiping.  

%The package is basically a redefinition of the \verb+\maketitle+ command.  The model was the New York Times---hopefully I haven't violated any copyright laws.  I also had to redefine the plain pagestyle.  It kept me busy for a few nights after work.  The rest is packages other people have written.

The twenty eighth day of May in the year of twenty and twenty one was a near summer day with the coolness and sunshine that makes a day worth spending outside.  



%\begin{window}[2,r,\includegraphics[width=1.0in]{atom.jpg},\centerline{The Atom}] The \verb+multicol+ package allows using multiple columns without starting a new page.  Using floats is not possible in a columns environment, however with the \verb+picinpar+ package, I can set a picture inside a block of text---just like you one you see here.  Isn't \LaTeX{} cool?
%And now we're just filling more space, and yet more space.  
%\end{window}

\closearticle


\headline{Adam Hughes News}
%This is just an example to fill up some space, but as long as I have your attention, I'll give some newspaper advice.
Don't talk about my friend like that. This is the time we have and we are not guaranteed any other moment. Do we get agitated about. Yeah because that's the way of the world and it feels good in the moment but we don't really know the true pleasure of also being in connection with God but once we get a taste of that then we realize like wait a minute all that stuff that he was talking about don't do this don't do that 10 Commandments this content of my meds that even the stuff that doesn't make sense it ends up making sense because we start.  

We start to see the process where oh wow that I didn't do this in this in this this this this this and then after a long time practicing those things if you see the connection where it goes all the way around and it flips through all these different dimensions and it's like a mystery of how it even happened but then you come back to the beginning and you just said yourself well I wanna follow God's commands because even though I don't know how it happened I know what the outcome is and I know what it does feel like to be closer to God. Keep on practicing that and it is a practice just like you asked me do the thoughts come to you and then you have to review them it's not like oh it's a one and done it's like oh after I do it and I do it a practice and get better and better at it and it just keeps getting better and better in the mysterious ways that happens I don't know what's going on but I sure do know I like the end result of being in communion with God and and how that relates totally to the things we experience in this world all the heavenly stuff like that. .Being with people who are also in that zone and it's like once you're with them and you experience that there's nothing in this world you trade for gold silver Ferraris you sacrifice at all and then you stop caring about the material stuff you stop caring about the wealth in the money and the food and the pleasures and we realize the time we have with people is the most important thing and so we cultivate it and we keep doing it and it's almost like.  

I've experienced where I think to myself and this is relevant to right now me and you where I know what's healthy for me to eat and I know like if somebody cared about me well and they had to fix my food will they fix the food the certain way and they feed it to me and it's like the food I don't even like all the sudden I wanna eat it just because it's healthy and just because I know God wants me to be healthy and like the stuff I used to eat when I was in that mindset like now I go back and I'm like.  And get water even though I really want to Coke how did I go to the the the the the gravy covered chicken fried steak in choose oatmeal to eat and it was like the reason was going all the way back to the time we have with people it's like the food doesn't matter it's the time that I'm getting to spend eating the food with the person I'm with it weird.  

%I suppose we could also show how an equation is type set:

%\begin{displaymath}
%x=\frac{-b\pm\sqrt{b^2-4ac}}{2a}
%\end{displaymath}

%and there you have it.  

%\lipsum[1-4]

\closearticle


\headline{Wall Street with Wallace}
Explain what just happened no offense but your last expose what 
you went through it went through this popularity contest relinquished 
I also how do you say preface and practicing and practicing this 
with no take no offense.  Don't let the sun so if you too much but 
your last expose I found found offensive because you just went 
through this whole popularity contest thing when popularity 
doesn't mean anything and then you said I'm hanging out with 
people because I want to be around popular people and then 
you said you want to be popular and have all these times.  
Away from the world and the worlds ways of doing that weird 
stuff.  And that's exactly what I just said was a little 
offensive because you saying that I'm shallow enough to to to 
to not introduce somebody say something because they're not 
popular in and all that stuff oh Michelle.  

Oh and scholarly mustard someone with the connection.  Temple here's 
a very good example that will take your expose and expose that it 
into a pancake do you think Zack is popular in a fancy with all his 
stuff has he ever met my sister no I just told you.  Because I was 
correct then I would've already introduced him to my sister squish 
pancake flat.  Geniuses that I'm usually right and I'm very wise.  
Play it with a hint – dump truck load of sarcasm.  
\closearticle

%\headline{Memorial Day 2021}
%
%\begin{window}[2,r,\includegraphics[width=1.3in]{Raising_the_Flag_on_Iwo_Jima,_larger_-_edit1.jpg},\centerline{Raising the Flag}] On Friday morning, February 23, 1945, four days after the Marines landed at Iwo Jima, Joe Rosenthal was making his daily visit to the island on a Marine landing craft when he heard that an American flag was being raised atop Mount Suribachi, a volcano at the southern tip of the island. Three of the men in the photo where later killed in fighting on the island.  Three of the men in the photo where later killed in fighting on the island.  Three of the men in the photo where later killed in fighting on the island.
%\closearticle

%\headline{December 23, 1944}
\headline{Battle of the Bulge}
23rd of December 1944 mission lost 10 B-26's and was awarded the 
Presidential Unit Citation for the 397th Bomb Group. Target: Eller 
Railroad Bridge
%Nord de Guerre

%https://b26.com/page/marauder_missions_december_23_1944.htm
%Marauder crews that flew on 23 Dec 1944
%
%Most of the following Marauder men were either MIA/KIA/POW
%
%41-31657, 387BG, 559BS, "MISSISSIPPI MUDCAT", Code TQ-W, Missions flown 149, MARC 11465
%23 Dec 44 shot down by Me109's, pilot killed by 20mm shell, went down immediately. 2.Lt's Vernon O Staub; Elgin F Scobell; William H Doyle; S/Sgt's Leonard Werner; Marvin P Geist; Sgt Richard F Wyman; Cpl Otto P Sicilano. (Staub, KIA; rest of crew, MIA)
%
%41-31842, 322BG, 451BS, "SIXOV US", Code SS-U, Missions flown 56, from 9 Sep 43 to 23 Apr 44 
%386BG, 552BS, "SIXOV US", Code RG-F, Missions flow 37, from 24 Jul 44 to 2 Dec 44 damaged at A-62 Reims/Champagne, repaired and reflew; pilot, Lt Danny R Duff, 23 Dec 44 crashed on take off, burst into flames. Maj. C.E. Hardy with 552BS crew. Maj. C.E. Hardy; 2.Lt Wesley E Brian; Capt Guy P Baird, Jr; 1.Lt Dean P Trigonides; T/Sgt Fred L Loftin; S/Sgt's Boyce W Ellington; Edward O Biegolman. (Hardy, Brian, Loftin survived; Baird, Trigonides, Ellington, killed; Regan, Biegolman died of injuries).
%
%41-3187, 387BG, 559BS "BOOGER RED II", Code TQ-O, Missions flown 131
%15 Sep 43 to 23 Dec 44 severely battle damaged, crashlanded on return; pilot, Lt. Theron G. Blackwell
%
%41-31896, 323BG, 453BS, "CIRCLE JERK" renamed "LOUISIANA MUD HEN", Code VT-G, Missions flown 121, MARC 11659
%
%21 Dec 43 to 23 Dec 44 flak in right engine over target, burst into flames, spun and crashed. 1.Lt's.James C. Bostick; Howard Detel; 2.Lt James P Hodges; S/Sgt's Albia Wiles; Robert Hehimer; A G Carrell (All KIA)
%
%41-31917, 387BG, 559BS, "DUCK BUTT II", Code TQ-M, Missions flown 111
%to 23 Dec 44 badly damaged by enemy fighters, crashlanded base, salvaged 23 Dec 44; pilot, 2.Lt.Warren R. Wade
%
%41-32058, 331CCTS, 1 Aug 43 Barksdale, La to 23 Dec 44 01.29hrs lost on navigational flight, contacted by Key Field control pilot reported he had 30 minutes of fuel left, Pilot given instructions for a instrument landing at Key Field. weather given as 400'ceiling, visibility one mile in light fog. Pilot let down under control, but with no power, brushed trees, crashed, utterly destroyed, no fire, 2 miles NW Bailey, Miss, and 10 miles N of Key Field. F/O's Arthur L Haugland; Howard B Beatty; Harold H Brush; Cpl's Stewart E Hazelgrove; John R Riesen; Lynn T Nelson (all killed in crash).
%
%42-95753, 320BG, 441BS, "TABOO" renamed "MY GAL", Code BN.08, Missions flown 115, MARC 11591
%19 Oct 44 flew it's 100th mission; 8 Jan 44 to 23 Dec 44 flak hit over the target cut a/c in two just aft of top turret, rolled over into 
%shallow spin, exploded on hitting ground. 2.Lt.Richard E. Dickey; 2.Lt Frank J Kuentz; 2.Lt Frank A Stadnick; Cpl Donald G Dickens; Pfc William Tipton; S/Sgt John W Lofton; Cpl Paul M Conant. (All MIA) 
%
%42-95798, 391BG, 574BS, "DRAGON WAGON", Code 4L-K, Missions flown 65, MARC 11673 
%2 Mar 44 to 23 Dec 44 hit by flak on bomb run, but no apparent damage, prior to reaching target attacked by fighters, dropped bombs, left engine shot out, fire raging in bomb bays crew bailed out, plane crashed. 2.Lts. Jack A. Haynes; John C Werner; 1.Lt William Greenough II; S/Sgts Harold W Weland; Edward C Schiffner; Wendell A Fetters. (Haynes killed when he bailed out with chute on fire, rest of crew POWs)
%
%42-95818, 391BG, 574BS, "LADY CHANCE", Code 4L-L, Missions flown 104, MARC 11677 
%23 Dec 44 hit in left engine by fighters, caught fire, dropped out of formation apparently under control. 1.Lt.Bartram L. Ryan, Jr; 2.Lts Claude Letzring; Allen V Rouse; S/Sgt's Thomas Netecke; Milton L Dean, Jr; Clinton B Trapp. (All MIA)
%
%42-95825, 391BG, 573BS, "EASY DOG 99", Code T6-B, Missions flown 108, MARC 11662 
%23 Dec 44 hit by fighters on pulling away from target, left engine smoking, right engine on reduced power, fire in radio compartment, crew bailed out. Capt.Joseph J.Boylan; 1.Lt Homer K Buerlein; Capt Norman S Dudley; Capt William L Smith; T/Sgt Allen E Adair; S/Sgts Robert V Vehr; Edward A Vich. (All POWs) 
%
%42-95838, 391BG, 574BS, Code 4L-B, Missions flown 100, MARC 11485
%23 Dec 44, fighter attack from rear destroyed tail turret, right engine on fire, right wing on fire, bombay on fire went down, exploded over edge of a small town. 1.Lt.Wilbur P. Stephenson; 2.Lts John J Kollar; Emil L Schwarz; Sgts Neel L Struwe; Arthur C Walstrom; Louis J Verdeal. (Stephenson, KIA, rest of crew POWs)
%
%42-95844, 391BG, 575BS, "MISS BEHAVIN", Code O8-D, Missions flown 96, MARC 11670 
%23 Dec 44 shot down by fighters, fire in bombay, fell out of formation spun, exploded and crashed. 2.Lts.William A. Kloepfer; John V Hultin; Edward E Wolf; S/Sgts Delmer L Haynes; Harold R Humble; James F Stevens. (Haynes, POW; rest of crew KIA). 
%
%42-95847, 391BG, 575BS, "SCRUMPTIOUS", Code O8-H, Missions flown 108 
%23 Dec 44 severely damaged by fighters, pilot Capt Breeseman; copilot Lt Curtis and two others severely wounded, controls taken over by Maj. Hershel S Harkins crashlanded back at base, plane salvaged 26 Dec 
%
%42-95855, 391BG, 575BS, "THE GRINNIN GREMLIN", Code O8-R, Missions flown 88 
%23 Dec 44 severely damaged by fighters, salvaged on return to base, salvaged 26 Dec 44; pilot, 2.Lt.William J. Kirton 
%
%42-95865, 391BG, 574BS, "SKYHAG", Code 4L-D, Missions flown 107, MARC 11664
%23 Dec 44 tail shot off by fighters, went down out of control. Capt. Clyde G.Brown; 2.Lts George E Bishop; Paul J Estrem; Sgts Robert L Vidler; Rudyard L Courtnay; S/Sgt Gene W Brillhart. (Brown, Bishop, Estrem, Vidler, KIA; Brillhart, POW; Courtney died of wounds) 
%
%42-95869, 387BG, 559BS, "THE FRONT BURNER II", Code TQ-F, Missions flown 93, MARC 11482 
%25 Feb 44 to 23 Dec 44 shot down by fighters, crashed at Junkerath, Germany. 2.Lts. Matthew J. Pusatari; Charles C Steward; Sgts John Robinson; Robert W Phillip; S/Sgt Eckard Munson; Sgt William M Mulligan; Cpl Howard Shweder. (All KIA)
%
%42-95878, 344BG, "WEARY LERA", to 22 May 44 transferred to 1st.PFF "WEARY LERA", Code IH-X, Missions flown 54, MARC 14626
%22 May 44 to 23 Dec 44 shot down by flak while leading 322.BG crashed in flames near Bohn, left engine on fire, three bailed out, two did not reach the ground alive. Capt. Wilbur G. Cox; 2.Lt Daniel P Winegar, Jr.; 1.Lt Ernest E Todd, Snr; T/Sgt's Ira L Mooney; Salvatore A Visi; S/Sgt Charles J Bohm. (Bohm, POW; rest of crew KIA)
%
%42-95932, 391BG, 575BS, "FIFINELLA" then "FIFINELLA DOG", Code O8-T, Missions flown 109, MARC 11551 
%23 Dec 44 shot down by fighters, fell out of formation with right engine ablaze. 1.Lts. Clark A. Tavener; Patrick H Wilkinson; 2.Lt Marc W Castle; T/Sgt Joseph W Wyne; Cpls John F McGettigan; Frank D Dick. (Wyne, Dick, POWs; rest of crew MIA) 
%
%42-96061, 394BG, 584BS, "HEAVENS ABOVE", Code K5-P, Missions flown 108, MARC 11402
%23 Dec 44 flak in left engine, unable to feather prop warning bell sounded, Folwell & Voorhis bailed out, prop windmilled, engine tore itself lose, went into uncontrollable spin, crashed. 2.Lt.Fred Reigner, Jr; 2.Lt Lester D Folwell POWs; S/Sgt Herman P Brueggeman; Sgts T J Bell; Jerome H Mendelson; Wilson E Voorhis, Jr. (Voorhis, Folwell, POWs; rest of crew KIA) 
%
%42-96144, 397BG, 596BS, "BANK NITE BETTY" Code X2-C, Missions flown 69, MARC 11483 
%23 Dec 44 hit by flak near top turret, exploded, broke into two pieces. 1.Lts. Charles W. Estes; William D Collins; Craig E Lewis; S/Sgt James P Negri; T/Sgt William E Epps; Sgt Bruno T Daskiewicz; Pfc Abraham J Korn. (All KIA)
%
%42-96148, 397BG, 596BS, "UNCLE BILLS FLAK HOUSE", Code X2-H, Missions flown 82
%23 Dec 44 damaged by fighters, landed on nosewheel only, repaired; pilot, Lt. Loy L. Julius
%
%42-96182, 397BG, 599BS, "MISS FURIE", Code 6B-K, Missions flown 64, MARC 11487 
%30 May 44 to 23 Dec 44 missing after attacks by enemy fighters. 1.Lt.Bernard F. Senart, Jr; 2.Lt Robert G Altman; Sgt Charles F Celeste; T/Sgt Eric C Swenson; S/Sgts Vito Portanova; Pasquale J Carino. (Altman, POW; rest of crew, KIA) 
%
%42-96201, 397BG, 599BS, "BLIND DATE", Code 6B-L, Mission flown 85 
%23 Dec 44 damaged in head on attack by Fw190's, fire started in bomb bay, copilot put out fire, engineers called to flight deck, misunderstood, bailed out, both engines stopped, then worked intermittently, pilot followed Group to crashland at base, nosed over, propeller dug into ground, tip broke off, hit copilot in head. 1.Lt. John H. Neu; 2.Lt J L Engle; S/Sgts HG B Moore; C D Quimby; W J Franton; T/Sgt J Reed. (Engle injured) 
%
%42-96203, 1st.PFF, "TERRIBLE TURK II", Code IH-S, Missions flown 35 
%23 Jun 44 to 23 Dec 44 landing accident at A-61 to service group, repaired; pilot, Capt Paul H. Jones
%
%42-96223, 1st.PFF, "CHERE AMIE", Code IH-O, Missions flown 39; the plane painted all over black; MARC 15002 
%21 Jun 44 to 23 Dec 44 shot down by fighters while leading 391st.BG, front of a/c badly shot up. Capt. Enoch G. Longsworth, 2.Lt Arthur E DeSaulniers, 1.Lt Phillip A Vogel; Capt William G Wilson, T/Sgt George R Winston; S/Sgt Robert H Aley (Longworth, Vogel, Wilson, DeSaulniers and Aley, KIA; Winston badly wounded but survived as POW)
%
%42-96252, 394BG, 585BS, Code H9-M, Missions flow 4, 20 May 44 to 30 May 44 transferred to 586BS, Code H9-M, Missions flown 66; 14 Jun 44 to 23 Dec 44 single engine, down wind landing, ran off runway, nosewheel collapsed, salvaged by 91 Air Depot Group 24 Dec 44; pilot, Lt. Harold S. Tuller
%
%42-96280, 397BG, 597BS, "BABY BUTCH II", Code 9F-B, Missions flown 1; 30 Jun 44 only, to service squadron; "BABY BUTCH II", Code 9F-T, Missions flown 36; MARC 11349
%10 Aug 44 to 23 Dec 44 missing after fighter attack, crew bailed out, crashed Waxweiler/Prum area. 2.Lts. Fred W. Marshall; Paul W McIntosh; S/Sgt James H Nichols; Cpl Harold M LaShelle; S/Sgt Mitchell Curry; Cpl Warren E Hampton (All POWs) 
%
%42-96309, 387BG 557BS, "SHIRLEY D", Code KS-G, Missions flown 27, MARC 10877
%6 Aug 44 to 23 Dec 44 shot down by flak near Bastogne, Belgium, crashed, exploded. Four parachuted and returned to base. 1.Lt. George E. Stith; 2.Lt's Jesse H Lutman; Ellis D Sutter; 1.Lt Lloyd L Hawsley; S/Sgt's Henry L Stevens; Hilbert G Swanson; Ephriam W Jackson; (Sutter, Hawsley, Swanson, Jackson, POWs; rest of crew MIA) 
%
%41-35010, 391BG, 574BS, "JINX/SNAKES REVENGE", Code 4L-A, Missions flown 46, MARC 11674 
%7 Jun 44 to 23 Dec 44 attacked by fighters, dropped out of formation losing oil from both engines, caught fire, went out of control, crashed 3km NW of Kolberg, Germany. 1.Lt. Dale Datjens; 2.Lt's Frederick T Kaye; Joseph M Blair; S/Sgt's Edward L Potocnik; Joseph K Kowalski; Joseph J Miller. (Potocnik, Kowalski, POWs; rest of crew killed in crash) 
%
%42-107577, 387BG, 559BS, "DUFFY" renamed "HOT GARTERS No.2", Code TQ-N, Missions flown 95, MARC 11403 
%15 Mar 44 to 23 Dec 44 shot down by fighters, severely damaged tail, rudder and elevators shredded, down in dive. 1.Lt. Wayne W. Church; 2.Lt Harold D Wagoner; Capt Jeff B Newman; T/Sgt James A Logero; S/Sgt's Frank C Porter; Paul D Duncan; Jimmy G Bort; Cpl Aubrey Waldron (Wagoner, Newman, Logero, Porter, Waldron, KIA; rest of crew POWs).
%
%42-107583, 344BG, 495BS, "SLEEPY TIME GAL", Missions flown 0; transferred without flying a combat mission with 344th.BG, 1.PFF, "SLEEPY TIME GAL", Code IH-E, Missions flown 51 (Codes IH-E issued but never applied to plane) 
%20 Apr 44 to 23 Dec 44 severely damaged by fighters, all gunners wounded, 300 holes in aircraft, right throttle shot wide open, airspeed out, electric trim out hydraulics shot out 5ft long x 22.5" wide hole in left wing; No brakes on landing at A-69 ran off end of runway into field, nosewheel collapsed. Salvaged. 1.Lts.Dale R. Bartels; Ray B Field; Bill Carls; Leonard Levin; S/Sgt's Curtis G Welborn; Gene Checkemain; Carl Ulery
%
%42-107597, 391BG, 574BS, "OLD SARGE" Code 4L-P, Missions flown 62; MARC 11669 
%9 May 44 to 23 Dec 44 hit by fighters, tail turret and gunner torn off in collision with enemy fighter both wings ablaze, exploded, crashed two miles SW Bonn, Germany. 1.Lt. Ralph H. Lesmeister; 2.Lts Albert O Stark; Carl B Pollock; S/Sgts Frank D Stanton; Clare H Oliphant, Jr; John H Stephenson. (Stephenson, KIA; rest of crew POWs) 
%
%42-107598, 387BG, 559BS, Code TQ-G, Missions flown 68; MARC 11646
%15 May 44 to 23 Dec 44 shot up by fighters in tail control surfaces, exploded in mid air. 1.Lt. William O. Pile; 2.Lt Robert W Ward; 2.Lt John J Walsh; 1.Lt John C Luce; 2.Lt Harold Ovis; T/Sgt George W Danner; S/Sgt Frank L Stasukt; Cpl Beldon W Hall; S/Sgt Wilfred L Becker. (Pile, Ward, Luce, Danner, MIA; rest of crew, POW).
%
%42-107614, 323BG, 454BS, "LADY LUCK III", Code RJ-H, Missions flown 12
%29 Sep 44 to 23 Dec 44 severe flak damage, crew bailed out over base, salvaged 23 Dec 44; pilot, Lt. William H. Eastwood
%
%42-107671, 391BG, 575BS, "SILVER DOLLAR", Code O8-L, Missions flown 75, MARC 11661
%2 May 44 to 23 Dec 44 shot up by fighters, incendiaries ignited in bombay, set bombay on fire, last seen with one engine burning other feathered, with wheels down, six crew parachuted out. 1.Lt. James F.Gatlin, Jr; 2.Lts Stephen V Biezis; John J Adair; S/Sgt Joe R Sanchez; M/Sgt William L Weissker; S/Sgt Milton E Cowart. (Gatlin; Biezis; Sanchez; Weissker, Adair, POWS; S/Sgt Milton E Cowart murdered on ground) 
%
%42-107672, 322BG, 450BS, "CHICKASA CHICK", Code ER-S, Missions flown 59
%15 May 44 to 23 Dec 44 cat AC damage, to service group; pilot 2.Lt. William L. Holmes
%
%42-107720, 391BG, 574BS, Code 4L-R, Missions flown 46, MARC 12222
%6 Aug 44 to 23 Dec 44 severely damaged by fighters bellylanded at base. repaired. 2.Lts Paul R Woods; Spangler; McNamara; Cpls Hancock; Lucero; Christenson. 25 Jan 45 to 10 Feb 45 hit by flak in right wing at 1515hrs, fuel tank in right wing, exploded, crashed in flames. 2.Lt's. Ted E. Martin; Seymour Smith; F/O William H Rochefort; Sgt's Frank J Spiezak; Robert L Silverston; Robert R Mann.(all kia) 
%
%42-107747, 391BG, 573BS, Code T6-T, Missions flown 51, MARC 11663; 13 May 44 to 23 Dec 44 shot down by fighters 3 miles E of Arweiler. Last seen dropping behind formation with both engines smoking, gear down, into spin, crashed. 1.Lt. Clayton S. Abraham; 2.Lt Verne H Bovie; S/Sgt Woodrow Wilson; T/Sgt Floyd B Lemon; Erik Christensen; Melvin E Murphy. (Bovie, POW; rest of crew, KIA) 
%
%43-34136, 397BG, 599BS, "RED DOG", Code 6B-U, Missions flown 24, MARC 13039
%23 Dec 44 vertical stabilizer shot away by fighters, did a complete roll, dropped out of formation with left engine on fire, seemed under control, into steep dive, crashed, burned, three chutes observed. 1.Lt. Charles T. Eiden; 2.Lt Robert L Dorn; S/Sgts Gerald D Jackson; William H Frisbie; T/Sgt Lewis W Sullivan; S/Sgt Charles T Berry. (Eiden, Dorn, Jackson, KIA; rest of crew, POW) 
%
%43-34139, 397BG, 599BS, "TOOT SWEET", Code 6B-Z, Missions flown 36, MARC 11986 
%15 Aug 44 to 23 Dec 44 shot down, circumstances unknown. 1.Lt. Robert E. McCarthy; 2.Lt's Charles F Abel; Harold T Ptaskiewicz; T/Sgt Clinton B Mathews; S/Sgts Ellis J Williamson; David E Hyre. (McCarthy, Ptaszkiewicz, Matthews, KIA; rest of crew, POWs) 
%
%43-34159, 397BG, 599BS, "HUN CONCIOUS II", Code 6B-J, Missions flown 16, MARC 11897 
%13 Aug 44 to 23 Dec 44 missing after attacks by enemy fighters. 1.Lt. Phillip C. Dryden; 2.Lt Robert F Stang; F/O Benjamin B Cummings; T/Sgts Elwood R Ahlgren; Stephen J Kish; Sgt Paul W LeFever. (Dryden died of wounds as a POW, Stang, Cummings, Ahlgren, KIA; Kish, LeFever, POWs). 
%
%43-34173, 323BG, 455BS, Code YU-J, Missions flown 7
%8 Nov 44 to 23 Dec 44 stalled and crashed into cleetrack on take off at Laon/Athies, salvaged 25 Dec 44. Pilot, Lt. Marcus R. McKinney 
%
%43-3417, 322BG, 452BS, Code DR-T, Missions flown 43
%1 Aug 44 to 23 Dec 44 hit by fighters at IP, damaged in right propeller and propeller, abandoned near Sedan (crew safe). Pilot, Lt. R.L. Eckhard 
%
%43-34185, 397BG, 598BS, "FWIGHTENED WABBIT", Code U2-A, Missions flown 32, MARC 11898
%13 Aug 44 to 23 Dec 44 missing after attacks by enemy fighters. 1.Lt. Lloyd D. Burns; 2.Lt's Alphonse W Westee; S/Sgts Richard W Whitney; Sgts E J Augusewicz; William C Owens; S/Sgt Robert D Peters (ALL POWs) 
%
%43-34221, 397BG, 598BS, "LIL' JAN", Code U2-L, Missions flown 14, MARC 11549 
%6 Sep 44 to 23 Dec 44 hit by fighters, rolled onto its back, split into two crashed near Auderath, Germany. Capt. Donald M. Stangle; 2.Lt Gordon H Wenborg; 1.Lt's Arthur E Coyne; Norman S Scherer; T/Sgt Harold W Perkins; S/Sgt's Joseph W Hejner; James H Hoots; Robert D Williams (All KIA) 
%
%43-34238, 319BG 438BS, Nov 44 transferred on conversion to B25's to BN.31, 17BG, 95BS ERMA BN.73, MARC 11590 
%23 Dec 44 shot down by Bf109, crashed near Achern. Six chutes seen. 2.Lt. Lane E. Spence; 2.Lt George E Williams; 2.Lt John R Steward; Sgt Peter J Yanko; Cpl Joseph A Renaldi, Sgt's Joe N Armstrong; Johnny C Gault. (Williams, POW; rest of crew, KIA)
%
%43-34260, 320BG, 444BS, "SUSAN", Code BN.97, Missions flown 28
%22 Aug 44 to 23 Dec 44 left tyre blew out on landing, ran off runway, ran through pile of gravel, strut wiped off at Y-14 Marignane salvaged by 304 Air Service Sqdn Jan 45. 2.Lt's Eugene R Channell; Frank J Catalano; S/Sgt Kenneth N Greear
%
%43-34309, 391BG, 575BS, Code T6-G, Missions flown 3; 20 Jul 44 to 6 Oct 44 damaged, to service squadron; transferred to 574BS, Code 4L-C, Missions flown 3, MARC 11651
%11 Dec 44 to 23 Dec 44 shot down by fighter attack from rear, tail gunner killed by 20mm shell in chest. 1.Lt.Clarence E. Michelson; Lt. Col Donald K Brandon; 2.Lts Ward C Smidl; Kenneth Messell; John R Peters; S/Sgts Jay S Troup; Eugene E Weibking; Robert D Buckley. (Michelson, Brandon killed in crash, Smidl; Messell, Peters, Troup, Weibking, POWs; Buckley killed in plane)
%
%43-34317, 322BG, 451BS, Code SS-H, Missions flown 17; 10 Nov 44 to 23 Dec 44 damaged cat AC, to service group; transferred to 452BS, Code DR-X8, 4 Mar 45 to wars end and scrapped.
%
%43-34361, 391BG, 574BS, Code 4L-M, Missions flown 23, MARC 11672
%2 Oct 44 to 23 Dec 44 fighters shot out left engine, elevators and radios, recovered tried to get back in formation, headed for home losing height steadily and under constant fighter fighter attack, rear bomb bay and ammunition on fire, dropped bombs, right engine on fire, lowered wheels, all but two gunners bailed out, crashed. 1.Lt. Paul N. Matus; 2.Lts William C Young; Earl E Cline; T/Sgt William C Swanson; S/Sgts Paul Raimonde; Harry T Stoeckel, Jr. (Swanson, Raimonde killed in crash)
%
%43-34423, 322BG, 450BS to 9 Nov 44 landing accident mechanical failure at Gt Saling, repaired; pilot, John L Egan 
%Transferred to 450BS, Code ER-V, Missions flown 5, MARC 11404
%9 Dec 44 to 23 Dec 44 hit by fighters, left engine on fire, tail and rudder damaged, spun down in flames over the target. 2.Lt's. James 
%F. Eckrich; William W Kuchinskas; 1.Lt Theodore A Bunn; T/Sgt Ralph M Wagner; Sgt's Alvin P Patterson; Peter R Olson (All POWs). 
%
%43-34430, 397BG, 599BS, "HUNCONSIDIOUS", Code 6B-B, Missions flown 4, MARC 11985
%1.Lt. William P.Cook; F/O Arthur J Lefavre; Sgt Eric M Honeyman; S/Sgts Ward C Swalwell, Jr; Frank G Lane, Jr; Maurice J Fevold (All KIA). 
%
%43-34434, 397BG, 599BS, "DRAGGIN' LADY No 3", Code 6B-C, Missions flown 18, MARC 11490 
%6 Sep 44 to 23 Dec 44 shot down by enemy fighters, crashed near Demerath. Capt. Mont F.Stephenson;2.Lt John L Grapes;1.Lts Robert J Kinney; Elmer R Borden; Laverne F Grundman; S/Sgt Lynn E Rose,Jr; T/Sgt William E Bower; S/Sgt's Harry W Watson; James V Galati (All KIA) 
%
%43-34440, 391BG, 575BS, Code O8-Q, Missions flown 15, MARC 11486
%13 Oct 44 to 23 Dec 44 right engine and wing set on fire and hydraulics shot out by fighters, bail out bell rung but did not work, copilot crawled through blazing bomb bay to warn crew in rear, then he and three other crew bailed out of rear, the selfless pilot crashing to his death 10 miles NW Bitburg, Germany. 2.Lt's.Donald N. Sharp; Raymond E Hedstrom; William D Hawkinson; Sgt's Edward H Von Castelberg; Glynn T Gilbert; Glenn E Doyal (Sharp, Hawkinson KIA). 
%
%44-67813, 322BG, 451BS, Code SS-P, Missions flown 2, MARC 12003
%Survived 23 Dec 44 missions, shot up on 2nd missions 1 Jan 45; flak in left engine and wing, burst into flames, left wheel, then left engine, then left wing fell off, into a spin, crashed. 1.Lt.Paul F. Michael; 2.Lts Robert L Weir; Vance I McCormick; Sgts Albert J Hands; A W Harriman; James C Rattan, Jr. (Michael, POW; rest of crew KIA). 
%
%44-67826, 391BG, 574BS, Code 4L-U, Missions flown 1, MARC 11660
%23 Dec 44 tail badly shot up by fighters, interphone out, both engines failed, pilot ,copilot and bomb bailed out, salvaged 27 Mar 45 1.Lt. Warren E.Grey; 2.Lt's Harry W Vaughn; Thomas D Crabb; T/Sgt's Edward S Smith; Daniel J Buckley; Ansel B Lay. (Grey, Vaughn, Crabb POWs; Smith, Lay, Buckley, KIA) 
%
%44-67881, 1.PFF, Code IH-W, Missions flown 1, MARC 14876/15984
%23 Dec 44 shot down by fighters leading 397BG. 1.Lt.Walter P. Garbisch; 2.Lt Herman L Wolfe; 1.Lt John R Berens; 2.Lt David B Lantz; S/Sgt's Francis J Boyd; Joseph H Shearon; Roger J Roy (ALL KIA)


\closearticle


\headline{1st Lt. William P. Cook}
%http://www.miaproject.net/mia-search-recoveries/hunconscious/

It was about 09:10 on this cold Saturday morning, December 23, 1944, 
when the last B-26 bomber left the base at Péronne France. The 4000hp 
of the two Pratt and Whitney R-2800 engines painfully lifted 
``Hunconscious”, the B-26G piloted by First Lieutenant William P. 
Cook of Alameda, California. Within minutes, she climbed to her 
cruising altitude and joined thirty-three other planes of the 
397th Bomb Group for the day’s mission, an attack against a railroad 
bridge spanning the Moselle River at Eller, Germany.

The first plane lost was ``Hunconscious”, Lieutenant Cook’s ship. 
Hit in the right engine, torque from the good motor pulled the 
machine into a snap roll. It then plummeted down toward earth. 
Cook and five other men perished.

Seconds after Flak hit Cook, a shell struck Lieutenant 
Charles W. Estes ship, ``Bank Nite Betty”. The deputy box 
leader and his crew described the incident during their 
post-mission debriefing. ``Three minutes after Malmedy, aircraft 
144 hit at about top turret and split in half. … No chutes seen.” 
All seven men aboard Estes’s aircraft died.

The group flew in a two-box formation and followed a pathfinder 
aircraft and three ``Window” aircraft. The latter carried 
no bombs and had the task of releasing strips of aluminum 
chaff to disrupt radar-controlled antiaircraft guns. The 
pathfinder carried four 500-pound bombs. The other planes 
each carried four 1,000-pound bombs. They flew for an hour 
over friendly territory before reaching the frontline city 
of Malmedy, Belgium. The airmen circled the city and took 
a course toward their objective. Enemy antiaircraft batteries 
immediately engaged the formation which was visible from the 
ground. At 10:20 AM, German gunners scored two hits. Sergeant 
McGinnis, a top-turret gunner, described the results. 
``Right off the bat, Flak got a ship in the right engine. 
She went over on her back. … Another blew in half.” 
Both victims came from Box II, the trailing box.

After losing Cook and Estes over Belgium, the 397th pressed ahead into German airspace and began the bomb run despite continued Flak. The aircraft jettisoned their explosives at 10:37 AM, bombing on the pathfinder’s mark. They hit the bridge. The formation made abroad left turn and headed toward home. Then disaster descended. Luftwaffe fighters pounced on the Americans. Aerial combat raged for ten minutes before P-47s intervened and broke up the attack. Flak and fighters claimed eight bombers over Germany. Only five planes returned to France without damage. Four of them were no longer airworthy and requiring months to repair. December 23, 1944, constituted the worst single-day losses for the 397th: ten aircraft downed, forty-five men dead, and twenty-one men prisoners of war. The group later received a Distinguished Unit Citation for destroying the Eller bridge.

After hostilities ended, the American Graves Registration Command accounted for thirty-nine of the dead. Lieutenant Cook and his crew were the only ones missing. The American Battle Monuments Commission later inscribed their names on the Wall of the Missing at the Luxembourg American Cemetery.

Decades later, during the 1990s, German aviation researchers began investigating the loss of Cook’s aircraft. The missing aircrew report contained little information regarding the circumstances of its disappearance. The report offered speculation that the aircraft fell somewhere in the geographic area where Luftwaffe fighters attacked the bombers. After years of work, they pinpointed nine 397th crash sites. Only one was in Belgium. This was the crash site of “Bank Nite Betty”, Estes’s aircraft. The plane had broken in two pieces while at altitude. The forward fuselage landed in a cow pasture southeast of Allmuthen, Belgium. The two German researchers thought the tail section had landed nearby, a belief that would eventually prove false.

During their research, they determined that antiaircraft fire had struck two 397th aircraft over Belgium, and both went down. The researchers knew that Estes had piloted one of them. Who piloted the other one? There was only one possibility—Cook. But where exactly did he crash?

In the autumn of 2006, the picture became clearer thanks to a German forestry worker affiliated with an organization called the Airwar History Working Group Rhine-Moselle. He had learned about the spot where the tail of Estes’s aircraft was supposed to have crashed. There was an impact crater. He searched in the vicinity and discovered over one hundred pounds of aircraft wreckage and shreds of U.S. Army Air Force clothing, including a leather fragment from the collar of a B-3 winter flying jacket. It bore a laundry mark, H-7489.

The Air War History Working Group identified the wreckage as being from a B-26. More importantly, they determined the laundry mark had no relation to anyone aboard Estes’s aircraft. The mark instead related to Cook’s bomb toggler, Sergeant Eric M. Honeyman (ASN 39037489) who also hailed from Alameda, California. This revelation led to an expanded search at the site in November. British-born aviation researcher Danny Keay joined the effort, and the group discovered more artifacts and two long-bone fragments.

Keay and the Germans contacted the U.S. Army Memorial Affairs Activity-Europe (USAMAA-E). In December 2006, two representatives from that organization arrived on site, conducted an investigation and took possession of the bone fragments and other artifacts. USAMAA-E deferred the case to the Joint POW-MIA Accounting Command (JPAC) for immediate action. Two JPAC historians visited the site in early 2007 and reported there were enough evidences “to warrant further investigation.”

Years rolled by. Despite the discovery of Honeyman’s flying jacket and meaningful bones, no recovery operation occurred. History Flight, a US non-profit organization, became involved in 2011 after developing a relationship with members of the 99th Division MIA Project. Bill Warnock, the project leader and his Belgian team mates Jean-Louis Seel and Jean-Philippe Speder had twenty years experience searching for missing soldiers of the 99th Infantry Division, having recovered twelve of them. The crash site lay on the periphery of the 99th sector and was well known to the Belgian-American group.

History Flight partnered with the group, and together they reconnoitered the crash site in June 2011 and found the area untouched since 2006. The task was going to be difficult. The site had been harvested years before and was now covered with a bush like vegetation. Moreover, the crater had been filled with branches and discarded logging cuttings. Seel and Speder extended the search on site in the Fall and unearthed many meaningful evidences, including equipment worn by crew members but no remains surfaced. This led to a larger operation in April 2012 which History Flight funded. Under the guidance of three American archaeologists, the consortium surveyed the site, pulled forest debris from the crater and drained its water for the first time in over sixty years. Buster, a cadaver dog attached to the Mammoth Lake Police Department in California, was also present to detect the potential presence of human decomposition chemicals. The team found more evidence identifying the specific B-26 model, a model consistent with Cook’s ship and not Estes’s. The team also recovered human remains and reported this to JPAC, USAMAA-E, and the Defense Prisoner of War/Missing Personnel Office (DPMO).

JPAC had now no other option than facing up to its responsibilities and dispatched forensic anthropologist Derek Benedix to the site. He examined the remains and recommended that JPAC conduct an immediate excavation of the crater and adjacent terrain. The first JPAC team arrived on site in May and began what proved to be a huge recovery operation. Beside the crater and its immediate surroundings, Honeyman dogtag editedthe debris field extended deep into a forest. A sizable quantity of human remains were eventually recovered as well as personal belongings, all consistent with the missing crewmen.The first element to surface was Sgt Honeyman’s dogtag. More were unearthed as the number of excavation Watch 01units augmented. Cook’s ID bracelet and dogtag, Swallwel and Lane’s dogtag and LeFavre’s bracelet and shoes were recovered. Unidentified but meaningful objects also came to light. Beside  a Sterling silver pilot wing badge in pristine condition, a torn officer’s cap badge,  English and French coins, a lighter, a fountain pen were among personal effects. Of particular significance an intact Elgin manufactured A-11 watch that had frozen the time of the crash. 10:25.

ong before thWard-C-Swalwel-Jre Belgian-American coalition started to work on this project, they had established contact with family members of Lieutenant Cook’s crew, including the sister of Ward Swalwell, a Chicago native and waist gunner aboard “Hunconscious”. When Jackie learned that her brother’s probable resting place had been located, she decided to visit the spot with her family. Escorted by 99th MIA Project members, they toured the site, met with JPAC team, and attended an improvised memorial service at the site. JPAC’s mission ended on September 17, 2013 after six recovery teams rotated on site. A total combined surface area of approximately 6400 square yards was excavated to an average depth of one foot.

In a memorandum dated July 22, 2014, JPAC confirmed the positive identification of the six crew members and notified the families.

Maurice J. Fevold was buried in Badger, Iowa, with full military honors on October 20, 2014. More than four miles of flags lined both sides of the street leading to Blossom Hill Cemetery for the soldier’s funeral procession.The Badger Fire Department also parked fire trucks along the route, and the Iowa Air National Guard provided a KC-135 plane that flew over the service.

William Parker Cook, the 27 year old aspiring surgeon and pilot of Hunconscious, had left his medical internship to enlist in the Army Air Forces. Just before going overseas, he had married his fiancee, Jean Swanson. He was buried in Mountain View Cemetery at Oakland, California, on October 26, 2014.

Frank G. Lane, Jr found his final resting place in Willoughby, Ohio, on May 2, 2015.

Eric M. Honeyman, although raised in California and enlisted in the Army Air Forces, was of Canadian origin. He was not granted to access US citizenship because he had not reached the legal age of 21. Eric’s parents, Bella and Eddy, had both donated their bodies to science, so there is no graves. The family decided to bring him home. Eric Honeyman was buried next to his grand parents in Trail, British Columbia, Canada, on June 22, 2015.

The last two crew members, co-pilot Arthur J. LeFavre and radio gunner Ward C. Swalwell, Jr were buried in Arlington National Cemetery, Virginia. LeFavre and a second casket containing unidentified osseous remains of the crew members were buried on August 18, 2015.  Ward Swalwell was buried next to them two days later, on August 20, 2015.

99th Division MIA Project members William C Warnock, Jean-Philippe Speder and Jean-Louis Seel,  History Flight, represented by Paul Schwimmer and his wife Audrey, joined the families and relatives of the six crew members. After a short funeral service held at the Old Post Chapel of Fort Myers, the flag draped casket with the unidentified remains of the six men was carried to the grave site by an artillery caisson drawn by six magnificent white horses. The horse escort and the caisson are reserved for officers and Medal of Play videoHonor recipients. LeFavre and Swalwell were placed in hearses and transferred to the grave site. After a brief eulogy, casket teams of the « Old Guard » folded the American flags in triangle with their unique, perfect and ceremonial drill. The chilling notes of « Taps » and a three rifle volley concluded the ceremony.

Abandoned in a Belgian forest for nearly seventy one years, Lt Cook and his crew found a final resting place befitting with the measure of their sacrifice.

Sources:
MACR 11985
Lt Cook, F/O LeFavre, S/Sgt Swalwel – IDPF’s
S/Sgt Lane, S/Sgt Fevold, Sgt Honeyman – IDPF’s
Preliminary report – Danny Keay

%https://b26.com/guestbook/2006.htm
%Date:
%11/23/2006
%Time:
%5:08 AM
% 
%Name: Axel Paul
%Bomb Group: 397th Bomb Group
%Bomb Squadron: 599th and 596th
%Years in service: 12/23/1944
%Location: Eifel area
%
%Comments: The crash site of a Martin B-26 „Marauder“ of the 397th Bomb Group was located in October 2006 – The plane was posted missing since 23. December 1944!
%
%Some weeks ago, a forest worker from Germany surprisingly found metal wreckage from a plane shot down in WWII and parts of clothing. He handled his findings over to the German “Arbeitsgemeinschaft Luftkriegsgeschichte Rhein / Mosel e.V.“, a group of air war historians working in that area documentation the air war 1939-45 since many years. It soon became clear that the parts belonged to a B-26 – one piece of clothing had still a part of a soldier’s personal number painted on it. This number confirms with the personal number of a crewmember of the B-26G-10-MA (Serial-No. 43-34430) from 599.Bomb Squadron / 397.Bomb Group, coded 6B-B. The 397.BG was stationed at A-72 Péronne (France) when on 23. December 1944 an attack against the Mosel bridges at Eller was launched. 34 “Marauders” took off with 1.000 lb. GP bombs to destroy the bridges. Reaching the area near Manderfeld close to the Belgian-German border, the ship of 1st Lt. William P. Cook and his crew (Ward C. Swalwell, Jr., Eric M. Honeyman, Arthur J. Le Favre, Maurice J. Fevold, Frank G. Lane, Jr.) ; got a flak hit and went down with its crew. The plane and its complete crew of six were posted missing from that day until 2006. It is planned to make a detailed research together with US officials to excavate the remains of the crew and the plane.
%
%The “AG Luftkriegsgeschichte Rhein / Mosel e.V.” asks to the veterans and friends of the 397. BG: Has anybody connections to the relatives of the crewmembers of this B-26?
%
%Also of interest: Has anybody connections to the relatives of the crew of B-26B-55-MA (Serial-No. 42-96144) from 596.BS / 397.BG (coded U2-C “Banknite Petty”). This “Marauder” under the command of 1st Lt. Charles W. Estes and his crew (William D. Collins, Craig L. Lewis, James P. Mecri, William E. Epps, Bruno T. Daszkiewicz, Abraham J. Korn) was also shot down in the same area the same day – the entire crew of seven was killed.
%
%We informed and work together with JPAC and USAMAA-E at Landstuhl. We do this also in several other cases too. We work close together with these officials since several years.
%We found there evidence to think that this could be the crash site of Cook. We stopped our activities there and talk with the owner to keep the site free of illegal treasure hunter. Don't be concerned in that thinks. We do this job here in Germany since 30 years. The reason for the posting was to find relatives, friends etc. to get in touch with them for more details and other information's.
% 
%Frank Güth


\closearticle


\headline{Capt. Mont F. Stephensen}


``A mormon from salt lake city, gosh... he was a great guy, 
he was killed later on, o gee... He was my buddy. You know, 
your lucky if you end up in life with one real buddy and I 
did that, and he was it." - 1st Lt. Jack L. Turner
Aircraft 4434


%https://b26.com/marauderman/mont_stephensen.htm
For the Fallen by Mrs. Lucille Grapes
``They shall not grow old, as we that are left grow old,
Age shall not weary them, nor the years condemn,
At the going down of the sun and in the morning,
We will remember them.''

%The Crew of the Draggin' Lady No. 3 who were
%Lost over Germany on December 23, 1944
%
%Capt. Mont F. Stephensen
%Lt. Elmer R. Borden
%Lt. Laverne F. Grundman
%Lt. Robert J. Kinney
%Lt. John L. Grapes
%T/Sgt. William E. Bower
%S/Sgt. Lynn E. Rose, Jr.
%S/Sgt. Harry H. Watson
%S/Sgt. James V. Galati
%
%from Mrs. Lucille Grapes



%https://b26.com/guestbook/2002.htm
%Date:
%5/5/2002
%Time:
%10:55:48 PM
% 
%Mont Fermore Stephensen BombGp: 397 Squadron: 599 Years: 1943-44 Location: Gardner Field, Calif. I am looking for information about my uncle Mont Stephensen. Lost in action December 23, 1944. The picture on Andy Andersons page is the same one that I have in my home. Thanks for your help. Mont

%Date:
%5/7/2002
%Time:
%8:17:10 AM
% 
%My brother, Capt. Mont Fermore Stephensen was a pilot, and squadron commander of a B26. We are trying to secure information to contribute to the Confederate Air Force Museum in Mesa, Arizona. Mont and his crew went down over Germany on Dec. 23, 1944. His plane was Draggin' Lady No. 3. 
% 
%Crew members were Lt. Elmer R. Borden, Lt. Laverne F. Grundman, St. Robert J. Kinney, St. John L. Grapes, T/Sgt. William E. Bower, S/Sgt. Lynn E. Rose, Jr., S/Sgt. Harry H. Watson and S/Sgt. James B. Galati.

\begin{window}[2,r,\includegraphics[width=1.3in]{43-34434-Page-4-copy.jpg},\centerline{Eller Railroad Bridge}] Mont Fermore Stephensen had enlisted in the United States Army Air Forces. Served during World War II. Stephensen had the rank of Captain. His military occupation or specialty was Pilot. Service number assignment was O-727795. Attached to 397th Bomb Group, 599th Bomb Squadron. \#43-34434 incident
%%https://www.honorstates.org/index.php?id=388508
%
%43-34430, 397BG, 599BS, "HUNCONSIDIOUS", Code 6B-B, Missions flown 4, MARC 11985
%1.Lt. William P.Cook; F/O Arthur J Lefavre; Sgt Eric M Honeyman; S/Sgts Ward C Swalwell, Jr; Frank G Lane, Jr; Maurice J Fevold (All KIA). 
%%https://b26.com/page/marauder_missions_december_23_1944.htm
\end{window}

\closearticle


\headline{1.Lt. Robert E. McCarthy}
1st Lt. Robert E. McCarthy is buried or memorialized at Plot E Row 4 Grave 38 Luxembourg American Cemetery Luxembourg City, Luxembourg. This is an American Battle Monuments Commission location.
On the 23rd of December 1944 he was the pilot of aircraft number 4139 with Abel, C. F. as co-pilot. The bomberdier for mission number 8 was 2nd Lt. Ptaskiewicz, H. T. with S. Sgt. Williamson, E. S. as engineer and T/Sgt. Mathews, C. B. as rear gunner and S/Sgt. Hyre, D. R. as aft gunner. 


%43-34139, 397BG, 599BS, ``TOOT SWEET'', Code 6B-Z, Missions flown 36, MARC 11986 
%15 Aug 44 to 23 Dec 44 shot down, circumstances unknown. 1.Lt. Robert E. McCarthy; 2.Lt's Charles F Abel; Harold T Ptaskiewicz; T//Sgt Clinton B Mathews; S//Sgts Ellis J Williamson; David E Hyre. (McCarthy, Ptaszkiewicz, Matthews, KIA; rest of crew, POWs) 
%Ellis J. ``Willie'' Williamson, aka ``Mr. Willie'' tells the story here on internet video: ``Gallery Furniture - Mattress Mack Sits Down With World War II Veteran Ellis J. Williamson'' 24 minute interview.
%https://www.youtube.com/watch?v=jYhApho19ro
Starting in Braintree, England, Williamson went to New Orleans for airplane mechanic school at Isaac Delgado trade school, then gunnery school in Ft. Barks, Florida. Oversees training was conducted in Florida and South Carolina. The engineer gunner is responsible for prepping the plane with gasoline and checking the tire pressure. After landing the plane they also stayed with the line until the gas tanks were full again. 

With 36 planes in a group.

Before this mission, the group had been grounded for about five days because of weather. This was the weather that the Nazis had waited for to begin and execute the battle of the bulge in the Ardennes forest . This was the first clear day that planes could fly and "America lost 178 planes that day and the Nazis lost their Air Force." The mission that day was to bomb the Eller Railroad Bridge in Eller Germany. First Lieutenant McCarthy's plane did not make it to the site, they were one of the first shut down near the town of Achim. After the B-26 got hit and the engine was on fire, Williamson recalls coming down out of the top turret where the pilot McCarthy was standing in the bomb-bay door and he told him to bail out. By this time the waist gunner and tail gunner had already bailed out. The plane was at about 8000 feet and one engine was on fire and the left landing gear was swinging underneath it. McCarthy got the bomb bay doors open and they had not dropped the bombs yet. McCarthy was standing in the bomb bay doors or radio room, and the copilot was flying the plane when Williamson left. This is the last he knew of any information. Williamson landed with his parachute and twisted his knee a little, but bigger trouble was standing over him, Nazis. They took him to a near by town and after that town was bombed by the allies with everyone taking cover in the basement of a building, he was taken to a prison camp to start his time there. ``From then on, it was just walking and stay ahead of the line.''

Williamson later talk to the copilot over the fence in the prison camp, and he said the pilot to come back and took control of the plane. The bombardier was on his hands and knees crawling out of the nose cone and the copilot bailed out, but the pilot and the bombardier went down with the plane. Bomberdier - Ptaskiewicz, H. T. 2nd Lt.
The waist gunner had been shot in the side from fire from the German fighter plane. The tail gunner said he saw him, later in the hospital after he got captured and said that he died in the hospital. Mathews. Rear Gunner - Mathews, C. B. T/Sgt. McCarthy and Ptaskiewicz are buried at the the Luxembourg American Cemetary.

%https://search.usa.gov/search?query=robert%20mccarthy&affiliate=dpaa&utf8=%26%23x2713%3B
\closearticle


``Seem to be the only men who can laugh and fight at the same time.'' - Winston Churchill remark about the U.S. soldier

``It is foolish and wrong to mourn the men who died, rather we should thank God that such men lived.'' - General George Patton
%http://www.polkcountytoday.com/yauponcover121419.html
%https://www.youtube.com/watch?v=63AoIEijbhs

%\end{window}

%\closearticle

\end{multicols}

\end{document}
